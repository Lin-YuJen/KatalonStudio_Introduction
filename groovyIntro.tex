\ifx\allfiles\undefined
\documentclass{beamer}
\usepackage{xeCJK}
\setmonofont[Mapping={}]{JetBrains Mono}
\setmainfont{JetBrains Mono}
\setsansfont{JetBrains Mono}
\setCJKmainfont{微軟正黑體}
\setCJKsansfont{微軟正黑體}
\usetheme[secheader]{Boadilla}
%\usepackage{ctable} % 表格
\usepackage{color} % 字體顏色
\usepackage{hyperref}
\usepackage{listings} % 程式碼區塊用
\lstset{
	language=Java,
	showtabs=false,
	showstringspaces=false
}
\begin{document}
\input{setting/command}
\fi
%----------------------------
\section{Groovy}
\begin{frame}
  \frametitle{Groovy 介紹}
  Groovy 是 Java 平台上設計的物件導向程式設計語言。
  擁有類似 Python、Ruby 中的一些特性,
  可以作為 Java 平台的 Scripting language 使用,
  Groovy 動態地編譯成執行於 JVM 上的 Java bytecode,
  並與其他 Java code 和 library 進行互操作。由於其執行在JVM上的特性,
  Groovy 可以使用其他 Java 編寫的 Library。~\\~\\
  Groovy 的語法與 Java 非常相似,
  大多數 Java code 也符合 Groovy 的語法規則,
  儘管可能語意不同。~\\
  \begin{center}
    \includegraphics[width=0.5\textwidth]{picture/Groovy-logo.png}
  \end{center}
\end{frame}
%----------------------------
\subsection{語法知識}
\begin{frame}[fragile]
  \frametitle{變數宣告與方法調用}
  變數可以用 def 關鍵字宣告,所有行句尾不必強制使用分號:
  \begin{block}{宣告一變數}
    def methodName = 'showMessage'
  \end{block}
  如果方法參數只有一個,可以省略其括號,即下列兩行等義:
  \begin{columns}
    \column{.49\textwidth}
    \begin{block}{輸出 Console 1}
      println('Hello World')
    \end{block}
    \column{.49\textwidth}
    \begin{block}{輸出 Console 2}
      println 'Hello World'
    \end{block}
  \end{columns}
  ~\\
  方法可以直接用方法名(String/GString)呼叫:
  \begin{block}{CustomKeywords 物件使用名稱為 verifyResponseErrorCode 的方法}
    CustomKeywords.'verifyResponseErrorCode' (response,'0000')
  \end{block}
\end{frame}
%----------------------------
\begin{frame}[fragile] % 含有 Code 的 frame 要加上 fragile
    \frametitle{Gtring and String}
    以下簡約表示 Gtring 與 String 的不同
    %\hyperlink{objectRepository}{\beamerbutton{objectRepository}}
    % -- Code block--
    \begin{block}{String:單引號}
\begin{lstlisting}
def string = 'this is a string'
\end{lstlisting}
    \end{block}

    % -- Code block--
    \begin{block}{GString:雙引號}
\begin{lstlisting}
def gstring = "hello, ${string}"
println gstring
\end{lstlisting}
    \end{block}
    
    如果使用 println 印出 gstring 變數會顯示 hello, this is a string
\end{frame}
%----------------------------
\begin{frame}[fragile]
  \begin{block}{Script:動態使用方法}
\begin{lstlisting}
def methodName = "showMessage"
def computer = new Computer()
computer."$methodName" "My name is John."
\end{lstlisting}
  \end{block}

  \begin{block}{Class 自訂類別}
\begin{lstlisting}
class Computer {
  void showMessage(def message) {
    println message
  }
}
\end{lstlisting}
  \end{block}
  
  \begin{block}{Console Output}
    My name is John. 
  \end{block}
\end{frame}
%----------------------------
\begin{frame}[fragile]
  \frametitle{List and Map}
  Groovy 有提供快速建立含有資料的 List 與 Map 的語法,使用中括號即可
  快速建立 ArrayList 與 LinkedHashMap 類別。~\\
  搭配 groovy.json.JsonBuilder 可以再轉為 Json 物件。
  \begin{block}{快速建立Map}
\begin{lstlisting}
def map = [
  "key":"value",
  "contractId":"CON0000001",
  "serviceIds":[
    "SVC00001",
    "SVC00002",
    "SVC00003"
  ]
]
\end{lstlisting}
  \end{block}
\end{frame}
%----------------------------
\subsection{實用類別}
\begin{frame}[fragile]
    \frametitle{JsonSlurper / XmlSlurper 類別}
    來自 groovy.json.JsonSlurper / groovy.json.XmlSlurper
    ~\\
    兩個類別可以協助轉換自 WS.sendRequest() 得到的 Response 物件中的內容
    \begin{block}{Sample}
\begin{lstlisting}
ResponseObject response =
  WS.sendRequest(findTestObject('CRMRequest'))
def xmlResult = 
  xmlSlurper.parseText(response.getResponseText())
def jsonResult = 
  jsonSlurper.parseText(xmlResult.text())
def contractId = jsonResult.CONTRACT_ID
\end{lstlisting}
    \end{block}
\end{frame}
%----------------------------
\begin{frame}[fragile]
    \frametitle{Sql 類別}
    來自 groovy.sql.Sql,可以快速建立資料庫連線
    \begin{block}{Sample}
\begin{lstlisting}
def database = Sql.newInstance(
  "jdbcUrl", "userName", "p6d", "driver"
)
def id = "CON01"
def sql = "SELECT * FROM USER WHERE ID = ${id}"
database.eachRow(sql){
  println it.getString("userName")
}
database.close()
\end{lstlisting}
    \end{block}
\end{frame}
%----------------------------
\ifx\allfiles\undefined
\end{document}
\fi
